
%% bare_conf.tex
%% V1.4
%% 2012/12/27
%% by Michael Shell
%% See:
%% http://www.michaelshell.org/
%% for current contact information.
%%
%% This is a skeleton file demonstrating the use of IEEEtran.cls
%% (requires IEEEtran.cls version 1.8 or later) with an IEEE conference paper.
%%
%% Support sites:
%% http://www.michaelshell.org/tex/ieeetran/
%% http://www.ctan.org/tex-archive/macros/latex/contrib/IEEEtran/
%% and
%% http://www.ieee.org/

\documentclass[journal]{IEEEtran}

\usepackage[utf8]{inputenc}
\usepackage{amsfonts}
\usepackage{amsmath}
\usepackage{amsthm}
\usepackage{amssymb}
\usepackage{amscd}
\usepackage{xspace}
\usepackage{mathtools}
\usepackage{wasysym}
\usepackage{ifsym}
\usepackage{bm}
\usepackage{upgreek}
\usepackage[backend=biber, sorting=none]{biblatex}
\addbibresource{sources.bib}

% Add the compsoc option for Computer Society conferences.
%
% If IEEEtran.cls has not been installed into the LaTeX system files,
% manually specify the path to it like:
% \documentclass[conference]{../sty/IEEEtran}


\begin{document}


\title{{\large INF-2990 Bachelor's Thesis in Informatics} \\ 
{\LARGE Verification of correctness of a concurrent hash map with TLA+}}


\author{Åsmund Aqissiaq Arild Kløvstad - Advisor: Håvard D. Johansen}
\date{\today}

% make the title area
\maketitle

% no keywords


\IEEEpeerreviewmaketitle

\textit{"The prevalence of programming errors has led to an interest in proving the correctness of programs"}

\section{Introduction}


\subsection{Thesis}
Shalev et al.'s concurrent hashmap can be implemented as a TLA+ specifcation and shown to be correct.
\\
The purpose of this specification is to increase confidence in model checkers by showing correspondence with the formal proof of Shalev et al.
\\
Additionally, the model will allow us to check other invariants and properties of the protocol like type correctness, liveness and the absence of deadlocks.
\subsection{Method}
\begin{enumerate}
    \item implement the protocol in TLA+
    \begin{enumerate}
        \item decreasing levels of abstraction
    \end{enumerate}
    \item check for the invariants proven by Shalev et.al using TLC model checker
\end{enumerate}

\subsection{Outline}
maybe an outline here

\section{Background}

\subsection{Concurrent programs}
\subsection{TLA+}
    \begin{enumerate}
        \item temporal logic of actions
        \item the Specification language
        \begin{itemize}
            \item syntax?
        \end{itemize}
        \item the TLC model checker
        \begin{itemize}
            \item state machines
            \item transition functions
            \item invariants
        \end{itemize}
    \end{enumerate}
\subsection{Shalev et.al's hashmap}
    \begin{enumerate}
        \item hashmaps in general
        \item resizing
        \item "lock free" (maybe in the concurrency background)
        \item buckets
        \item split-ordering
        \item dumy nodoes
    \end{enumerate}

\section{The Specification}
Low on details for now
    \begin{enumerate}
        \item assumptions and abstraction
        \item step size
        \item transition function
        \item specifying invariants
    \end{enumerate}

\section{Results / evaluation}
    \begin{enumerate}
        \item what did we prove?
        \item is that the same as Shalev et.al claims?
        \item was TLA+ useful?
    \end{enumerate}

\section{Conclusion}
    \begin{enumerate}
        \item what was done?
        \item what did we prove / find out?
        \item further work? maybe its own section
    \end{enumerate}


\printbibliography{}

% that's all folks
\end{document}


