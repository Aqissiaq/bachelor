\documentclass{uit-thesis}

\usepackage{enumitem}
\usepackage[backend=biber, sorting=none]{biblatex}
\addbibresource{sources.bib}

\newcommand{\Z}{\mathbb{Z}}
\newcommand{\N}{\mathbb{N}}



\begin{document}

\title{Formal Verification of a Concurrent Hashmap}
% \subtitle{Subtitle}% Note: this is optional, and may be commented out
\author{Åsmund Aqissiaq Arild Kløvstad}
\thesisfaculty{Faculty of Science and Technology \\ Department of Computer Science}
\thesisprogramme{INF-2990 Bachelor's thesis in Informatics \today{}}

\maketitle

\frontmatter

\tableofcontents

\mainmatter

\chapter{Introduction}
\section{Thesis}
\section{Method/scope}
\section{Outline}

\chapter{Background}
% \vfill{}
\paragraph{Model Checking: Algorithmic Verification and Debugging\cite{Clarke2009}}
In the Turing Lecture by the winners of the 2007 Turing Award, Edmund Clarke, Allen Emerson and Joseph Sifakis they describe the develpment and use of model checkers as a verification method for computer systems. Previous efforts to prove correctness had been focused on formal proofs which have three key shortcomings:
\begin{enumerate}
    \item they require human ingenuity,
    \item they are difficult to work with in concurrent and distributed systems,
    \item they scale poorly with system size and complexity.
\end{enumerate}
Instead, they propose algorithmic model checkers.
\\\\
With this method a Temporal Logic is used to specify the correct behavior of a system and the model checker verifies that this behavior is not violated by exploring the state space of the model. Importantly, such model checkers produce a counter example -- an example of incorrect behavior -- which makes debugging and correcting the system easier.
Key properties of a temporal logic are \textit{expressiveness} and \textit{efficiency}.
\\\\
Model checking also scales poorly with system complexity, so several techniques are introduced to deal with "state space explosion"
\begin{itemize}
    \item symbolic checking of ordered binary decision diagrams
    \item isolation of independent events in concurrent systems
    \item bounded checking by solving SAT
    \item reduce state space by increasing level of abstraction
    \begin{itemize}
        \item if counterexamples are found a lower abstraction level is needed, but "good" properties hold through abstraction mappings
    \end{itemize}
\end{itemize}


\paragraph{How Amazon Web Services Uses Formal Methods\cite{Amazon2015}}
Amazon's AWS services are all underpinned by large and complex distributed systems. This is necessary for high availability, growth and cost-effective infrastructure. Traditionally these systems have been tested by savvy engineers who know what to test and look for. However, some errors are very rare and will very likely slip through such testing. To catch these errors they employ model checking (with TLA+).
\\\\
The PlusCal or TLA+ specifications work as a tool to bridge the gap between design and implementation. Designs are expressive, but inprecise while the implementation is precise, but hides overall structure. Through a choice of abstraction level, specifications can bridge this gap and provide both. An expressive specification also provides useful documentation of the system.
\\\\
The key benefits of model checkers at Amazon are:
\begin{itemize}
    \item a precisely specified design helps make changes and optimizations safely. This usage improves system understanding.
    \item they are faster than formal proofs
    \item a correct design and the understanding the specs provide promote better, more correct code.
\end{itemize}

\section{Temporal Logics}
\paragraph{The Temporal Logic of Programs \cite{Pnueli1977}}
In The Temporal Logic of Programs\cite{Pnueli1977}, Amir Pnueli proposes a unified approach to the verification of both sequential and concurrent programs. His work seeks to unify approaches to to both, while also presenting a system that emulates the design intuition of programmers. The key concepts in this work are \textit{invariance} -- which covers partial correctness, clean behavior, mutual exlusion and deadlock freedom -- and \textit{eventuality} -- which generalizes these notions to cyclic programs and provides a special case of total correctness.
\\\\
A dynamic discrete system is generalized as a three-tuple $\langle S, R, s_0 \rangle$ where $S$ is the set of possible states, $R$ a transition relation, and $s_0$ the initial state of the system. In order to make later constructions easier we further specify
$$s = \langle \pi, u \rangle$$ where $\pi$ is the control component specifying the location in the program and $u$ is the data component describing the state of any variables and data structures, and
$$R(\pi, u) = N(\pi, u) \land T(\pi, u)$$ where $N$ describes the control flow and T the change in data such that a step in the execution may be described by
$$R(\langle \pi, u \rangle , \langle \pi', u' \rangle) \iff \pi' = N(\pi, u) \land u' = T(\pi, u)$$
To reason about concurrent programs we let states have multiple control components $s = \langle \pi_1, \pi_2,...,\pi_n, u \rangle$ and randomly choose one control component to update in each step. Finally we let $X$ be the set of all reachable states for the system. A predicate $p(s)$ is \textbf{invariant} if $p(s)$ is true $\forall s \in X$
\\\\
We can now start to define useful properties of the systems described in this way.
\begin{itemize}[label={}]
    \item \textbf{Partial correctness} is the claim that given the correct input, a program produces the correct output. We let $\phi(x)$ be the statement "reaching the end state $\implies$ (correct input $\implies$ correct output)". Partial correctness is equivalent to saying $\phi$ is an invariant.
    \item \textbf{Clean execution} means the program does not behave illegally, i.e it does not access illegal memory locations or divide by zero. We may define these restrictions as a predicate to make clean execution equivalent to this predicate being invariant.
    \item \textbf{Mutual exclusion}. Given a critical section $C$, mutual exclusion of the processes $\pi_1$ and $\pi_2$ is described by the invariance of the predicate $\lnot(\pi_1 \in C \land \pi_2 \in C)$.
\end{itemize}
In addition to these properties we wish to reason about \textit{temporal} implications. We let time be described by a $t \in \N$ and $H(p,t)$ denote the value of the predicate $p$ at time $t$.
We then introduce the temporal operator $p \rightsquigarrow q$ to mean $p$ eventually leads to $q$, or formally:
$$p \rightsquigarrow q: \forall t_1 \exists t_2\ s.t\ t_1 \le t_2,  H(p, t_1) \implies H(q, t_2)$$
For all times $t_1$ there is a later time $t_2$ such that if $p$ holds at $t_1$, $q$ will hold at $t_2$.
Armed with eventuality we can define temporally useful properties of systems.
\begin{itemize}[label={}]
    \item \textbf{Total correctness} is stronger than partial correctness because it also requires that the program reaches an end state. We can express total correctness as $\langle \pi = l_0, u = \phi\rangle \rightsquigarrow \langle \pi = l_m, u = \psi\rangle$ where $\phi$ denotes correct input, and $\psi$ denotes correct output and $l_0$, $l_m$ are the start and end labeles of the system, respectively.
    \item \textbf{Accessibility} is the guarantee that some segment $S$ of a program can be reached. It can be expressed by $\pi = l_0 \rightsquigarrow \pi \in S$
    \item \textbf{Responsiveness}. It is often desirable that some request $p$ will be met by a response $q$. We call this responsiveness and describe it by $r \rightsquigarrow q$.
\end{itemize}

With all the notation out of the way, define axioms and inferece rules. Then introduce F and G.


\section{TLA+}
\section{Split-Ordered List Hashmap}

Maybe a description of hashmaps in general here to set up the later use of table, list, key etc. or is that assumed knowledge?
\\\\
Shalev et al.\cite{Shalev2006} present the first lock-free extensible hash table implemented using only loads, stores and atomic Compare and Swap (CAS).
\\\\
Hashmaps are a key building block in many important systems [citation needed], but are difficult to implement concurrently. In particular, the resizing (extending) of the table is difficult to do atomically because at the very least a node must be moved from one list to another. In order to avoid conflicts and loss of data in this process some overhead is required which impacts performance.
\\\\
The key insight of Shalev et al. is to flip the process upside down. Instead of moving nodes between buckets, they suggest moving the buckets among a statically ordered list of nodes. This requires an ordering of the list in which a bucket can always be split into two new buckets while their contents remain correct. A node should always reside in the bucket corresponding to its key $\mod 2^i$ where $2^i$ is the current size of the table. 
\paragraph{Split-Ordered Lists}
are introduced to solve this problem. By sorting the keys according to their reversed binary representation they obtain a list which can be always be split into buckets $\mod 2^i$. This is because such an ordering corresponds to difference in the keys' $i$th least significant bit, which is equivalent to having a different remainder $\mod 2^i$.
\\\\
In order to deal with the problems caused by removing nodes pointed to by hash table entries dummy nodes with the bucket value are introduced. These nodes signify the start of a bucket and are recursively initialized when an item is inserted into an unitialized bucket. To distinguish dummy nodes from regular nodes in the list, regular node keys have their most significant bit set to 1 before being reversed.

\backmatter

\printbibliography{}

\end{document}