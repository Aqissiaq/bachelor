
%% bare_conf.tex
%% V1.4
%% 2012/12/27
%% by Michael Shell
%% See:
%% http://www.michaelshell.org/
%% for current contact information.
%%
%% This is a skeleton file demonstrating the use of IEEEtran.cls
%% (requires IEEEtran.cls version 1.8 or later) with an IEEE conference paper.
%%
%% Support sites:
%% http://www.michaelshell.org/tex/ieeetran/
%% http://www.ctan.org/tex-archive/macros/latex/contrib/IEEEtran/
%% and
%% http://www.ieee.org/

\documentclass[journal]{IEEEtran}

\usepackage[utf8]{inputenc}
\usepackage{amsfonts}
\usepackage{amsmath}
\usepackage{amsthm}
\usepackage{amssymb}
\usepackage{amscd}
\usepackage{xspace}
\usepackage{mathtools}
\usepackage{wasysym}
\usepackage{ifsym}
\usepackage{bm}
\usepackage{upgreek}
\usepackage[backend=biber, sorting=none]{biblatex}
\addbibresource{proposal_sources.bib}

% Add the compsoc option for Computer Society conferences.
%
% If IEEEtran.cls has not been installed into the LaTeX system files,
% manually specify the path to it like:
% \documentclass[conference]{../sty/IEEEtran}


\begin{document}


\title{{\large INF-2990 Bachelor's Thesis in Informatics} \\ 
{\LARGE Verification of correctness of a concurrent hash map with TLA+}}


\author{Åsmund Aqissiaq Arild Kløvstad - Advisor: Håvard D. Johansen}
\date{\today}

% make the title area
\maketitle

% no keywords


\IEEEpeerreviewmaketitle

\section{Introduction}
As early as 1977 Leslie Lamport noted the difficulty and importance of writing correct programs in a concurrent setting\cite{Lamport1977}. Since then concurrency has become increasingly important due to the power wall and the high level of parallelism in modern CPUs\cite{Tanenbaum2014}.

It is difficult for programmers to reason about concurrent execution and this leads to bugs which are often subtle and difficult to reproduce\cite{Beschastnikh2016}. It is useful, then, to be able to prove the correctness of such programs. The main approaches to this are
\begin{itemize}
    \item exhaustive testing,
    \item formal proofs, and
    \item model checking.
\end{itemize}

Exhaustive testing is often time consuming and may miss rare or unexpected errors. Formal proofs prove correctness for all cases, but may be outside the expertise of many programmers. A model checker is used to check the specification of a program for correctness by viewing it as a finite state machine and searching the possible state space\cite{Clarke2009}. It is of interest whether such an approach can be as good as a formal proof while still being practical for programmers.

% Correct programs are important

% This is difficult in concurrent settings

% How to prove correctness?
\section{Project Description}
This project will use the TLA specification language and the TLC model checker to verify the correctness of a concurrent data structure. The model checker can then be used to check the data structure for other properties.

The data structure in question is a lock-free concurrent hashmap\cite{Shalev2006}. This structure is formally proven to be correct -- specifically Shalev et.al show that the keys are sorted, that each key points to the correct bucket, and that the basic operations \textit{insert}, \textit{find} and \textit{delete} behave correctly -- so using a model checker to confirm will strengthen both trust in the formal proof and in model checkers' usefulness.

Additionally, once the specification is written the model can be used to check other properties of the data structure such as liveness or type correctness.
% \begin{enumerate}
%     \item learn about TLA+
%     \item specify a data structure (Shalev's hash map)
%     \item use TLC to prove stuffs
%     \begin{itemize}
%         \item correctness (keys sorted, correct buckets)
%         \item liveness
%         \item other?
%     \end{itemize}
%     \item draw some unfounded conclusions about model checkers probably
% \end{enumerate}


\section{Suggested Reading List}
The following works are recommended for completing this course.
\newrefsegment
\nocite{Clarke2009}
\nocite{Lund2019}
\nocite{Shalev2006}
\nocite{Lamport_specifying_2002}
\nocite{Amazon2015}
\nocite{Lamport_video_2019}
\printbibliography[heading=none,segment=1]

\noindent
Note that this reading list should not be considered final. The student is
expected to research the given topic and expand this
list for the final report.

\section{Deliverables}
The final hand-in will consist of
\begin{itemize}
\item a thesis describing the work done and the results of model checking
\item a specification of Shalev et. al's hashmap written in PlusCal
\end{itemize}
\noindent
The final deadline is June 1st at 23:59 and the project will be graded on an A-F letter scale. The project is worth 10 credits.

\section{Timeline}
\begin{itemize}
    \item Week 7 Project start
    \item Week 9 Thesis first draft
    \item Week 12 Specification first draft
    \item Week 14 Mid-project eval
    \item Week 20 Final draft hand-in of thesis and specification
    \item Week 22 Final hand-in
\end{itemize}

%\section{Deliverables}
%\begin{itemize}
%    \item A written thesis 
%    \item Formal models of an authorization protocol and a Diggi component
%    \item A machine-checkable proof of the protocol
%    \item Source code for the authorization service implementing the protocol
%    \item Instructions on how to set up and run the authorization service, check the models and verify the proofs
%\end{itemize}
% References. Use IEEE as citation style, and load references from report.bib.

\printbibliography{}

% that's all folks
\end{document}


